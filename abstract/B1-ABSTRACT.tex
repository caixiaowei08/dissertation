\vspace{-2.5cm}
\chapter*{\zihao{2}\heiti{摘~~~~要}}
\vspace{-1cm}

20世纪70年代末,随着智能机器人学、传感技术和分布式系统的快速发展,自主移动机器人(autonomous mobile robots)逐渐步入了人类的生活当中,尤其是在工业生产、空间探索和军事等领域得到了广泛的应用。多自主移动机器人协作对未知空间探索问题是近期该领域的一个研究热点。

移动算法是多自主移动机器人协作完成空间探索问题的核心,自主移动机器人通过视觉传感器获取周围环境中其他机器人的位置信息,将位置信息与移动算法进行匹配,做出对应的移动决策。未知空间模型由连续二维欧几里德空间模型,逐渐演化成为有限位置的离散空间模型。在离散空间模型中,机器人有三种调度模型:完全同步模型FSYNC(Full-synchronous model)、 半同步调度模型SSYNC(Semi-synchronous model)、 异步调度模型ASYNC(Asynchronous model)。验证不同调度模型下,移动算法是否满足空间探索性质大多数使用的是手动推演的方法,手动推演的方式不仅过程冗长复杂,而且演算过程中容易出现错误。对于异步调度模型中,由于存在使用过时环境位置信息作为移动决策依据的情况,根本无法使用手动推演的方式进行验证。

而形式化验证方法因其自动性、高效性和严谨性,弥补了手动推演的不足之处,逐渐被用于各种自主移动机器人算法的验证。近期,B Bérard使用DVE形式化语言对机器人移动算进行了建模,并使用DiVinE和ITS形式化工具验证了机器人移动算法模型是否满足永恒探索性质。另外,Ha等人使用 Maude重写逻辑形式化语言实现移动机器人永恒探索算法的验证。已知的这些方法多针对某
种特定的调度模型进行建模,并通过人为设定一个具体的初始状态进行验证.然而自主移动机器人算法
中系统的初始状态多是未知的。

本文提出自主移动机器人符号模型检测方法,该方法不依赖某个具体的初始状态,且适用于不同的调度模型.同时,借助符号模型检测的高效性,有效避免状态爆炸问题.利用nuXmv 符号模型验证工具对机器人探索算法在三种调度模型:完全同步模型、半同步模型SSYNC、异步模型进行建模并利用LTL公式定义算法的永恒探索性质,最终实现移动算法的形式化验证.验证结果表明在假设初始状态未知的条件下依然可验证性质不被满足并找到反例.同时,实验数据表明了符号模型检测对自主移动机器人算法形式化验证的可行性与高效性.

\hspace{-0.5cm}
\sihao{\heiti{关键词:}} \xiaosi{nuXmv,移动机器人,空间探索,符号模型检测,形式化验证}

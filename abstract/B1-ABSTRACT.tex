\vspace{-2.5cm}
\chapter*{\zihao{2}\heiti{摘~~~~要}}
\vspace{-1cm}

随着物联网技术的发展,自主移动机器人(autonomous mobile robots)在网络中的作用也日益重要,利用形式化的方法验证自主移动机器人行为的正确性逐渐成为新的研究热点.当前主要的验证方法多以初始状态已知为前提且面临状态爆炸问题.本文以自主移动机器人空间永恒探索算法为例,提出自主移动机器人符号模型检测方法,该方法不依赖某个具体的初始状态,且适用于不同的同步模型.同时,借助符号模型检测的高效性,有效避免状态爆炸问题.利用nuXmv符号模型验证工具对机器人探索算法在三种同步模型:完全同步模型FSYNC(Full-synchronous model)、半同步模型SSYNC(semi-synchronous model)、异步模型ASYNC(Asynchronous model)进行建模并利用LTL公式定义算法的性质,最终实现算法的形式化验证.验证结果表明在假设初始状态未知的条件下依然可验证性质不被满足并找到反例.同时,实验数据表明了符号模型检测对自主移动机器人算法形式化验证的可行性与高效性.


\hspace{-0.5cm}
\sihao{\heiti{关键词:}} \xiaosi{nuXmv,移动机器人,空间探索,符号模型检测,形式化验证}

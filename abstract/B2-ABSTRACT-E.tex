\newpage
\vspace{-1cm}
\chapter*{\zihao{-2}\heiti{ABSTRACT}}
\vspace{-0.5cm}

With the development of Internet of Things technology, the role that autonomous mobile robots play in the network is becoming increasingly important. Formal verification of the correctness of autonomous mobile robots has become a new research topic. Most of the exising approaches either suffer state-explosion problem or rely on concrete initial states, which however are usually not undefined and have to be enumerated to make verification complete. In this paper, we propose a symbolic model checking approach to the formal verification of a typical automonous mobile robot system called mobile robot perpetual exploration system. For the symbolicity feature of the model checking, the verifcation does not rely on specific initial states and meanwhile state-explosion problem can be avoided. We use the state-of-the-art symbolic model checker nuXmv to verify the mobile robot perpetual exploration algorithm under three different scheduling modes called full synchronous model (FSYNC), semi-synchronous model (SSYNC) and asynchronous model (ASYNC). Experimental results show that even without providing specific initial states a counterexample can be found in our approach for the perpetual exploration property, which concides with the existing verification result which is obtained by model checking with specific initial states. Meanwhile, the experimental data shows the feasibility and efficiency of symbolic model checking in the formal verification of autonomous mobile robot systems.

\hspace{-0.5cm}

{\sihao{\textbf{Keywords:}}} \textit{nuXmv,\, LTL,\, Mobile robots,\, space exploration,\, symbolic model checking
}

































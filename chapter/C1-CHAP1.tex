\chapter{绪\hskip 0.4cm 论}


\section{研究背景与意义}

随着网络时代的发展,当代人类对无线技术的需求愈发增加,不但要求像3G技术的移动性和快速切换性,而且希望能得到类似宽带网络一样的高带宽与传输准确性。由此,基于IEEE802.16m\cite{Ieee2011IEEE}标准的移动WiMAX\cite{Andrews2007Fundamentals}网络应运而生,WiMAX是一种城域无线宽带技术,因此被认定为即将成为下一代4G技术标准。WiMAX能够覆盖更大的频率范围以及能够实现更广泛的传输距离,具有更高的可拓展性以及支持QoS传输质量保证。由于无线传输媒体十分容易遭受到外部攻击的特性,安全性问题一直在无线网络消息传输中处于至关重要的位置。

在IEEE802.16m标准的设计中,一开始便考虑到如何保证无线传输技术的安全性问题。WiMAX在物理层和MAC层利用很多先进技术和系统机制保护信息传输的机密性和完整性,以及抵御不同类型的网络安全攻击。IEEE802.16m标准在MAC层的安全子层中定义了无线宽带连接服务的隐私保护功能,用于实现基站与客户端间建立安全连接。安全子层中主要包括用于加密数据包的封装协议和密钥管理协议PKMv3协议\cite{李凤海2006无线城域网安全子层分析与研究}。封装协议包括一系列的密码学套件和MAC数据包的转发规则,PKMv3协议提供了基站与手机端间的身份验证以及密钥分发,实现站点间的密钥材料同步共享。

PKMv3协议是WiMAX技术中的第三代密钥管理协议,在前两代的基础上进行了安全性改进,增加了对消息的分级保护,并在加密运算中采用更加安全的AES对称加密算法,并支持EAP(Extensible Authentication Protocol)的认证方式和CMAC(Cipher-based Message Authentication Code)保护的消息验证。目前针对PKMv3协议的研究仍然非常之少,因此分析与检验该密钥管理协议的安全性质十分关键。安全协议的设计目标为实现身份鉴权双方的认证性,关键信息的机密性,传输的完整性,计算资源的可用性以及对目标服务的访问控制。评估并分析安全协议的安全性标准为在该安全协议所处的环境中,协议设计目标中的安全特性是否有可达性缺陷或是否会被人为破坏。

 安全协议的形式化方法经过长期的发展,目前已经实现了比较完善的理论体系。协议的形式化分析方法的基本原理是采用逻辑模型,通过有效的程序分析协议系统,判断系统所满足的特定条件以此证明某种协议性质。而传统的人工分析与验证形式化模型,计算效率不高而且会带来因人为疏忽带来的验证结果不准确性。通过自动化协议验证工具,能够很大提高协议分析的工作效率与准确性。

     安全协议的形式化分析的理论体系分为三种:基于模态逻辑技术的形式化推理,将协议中的对象及其操作实现逻辑抽象,并通过演绎推理实现主体知识的搜集与推导,从而验证系统是否能够满足某些安全属性;基于模型检测技术的形式化分析方法,将安全协议模拟为分布式系统,主体参与协议过程的行为引发系统局部状态的改变,各局部状态的变化造成整个系统全局状态的改变,检测可达状态所满足的安全属性;定理证明方法,将安全协议定义为定理系统,通过定理证明实现协议安全性质的检测\cite{鲁来凤2012 安全协议形式化分析理论与应用研究}。

目前形式化方法由于其成本代价小以及高效准确的特性,正逐渐被广泛接受和应用。基于重写逻辑的Maude\cite{Clavel2007All}语言是一种高级编程以及形式化建模语言,可以实现分布式以及大型系统的建模。通过Maude工具实现对安全协议分析验证,已有很多学者做出相关成功案例。Maude具有用户友好和直观的形式化语义以及含义,可用于设计与检测系统或者协议模型。Maude同时支持面向对象以及模块化的规格说明和分析。同时能够提供高效以及稳定的自动化仿真和建模工具,实现系统模型的形式化验证。

     相比而言,C、Java等普通的非形式化语言在描述协议时,对协议通信过程的规格说明太过复杂,可读性较差,对协议性质无法实现清晰的定义,描述语言容易产生歧义。同时,这些编程语言必须通过线程来模拟系统的并发执行,无法使用高效的规格说明描述分布式系统。并且也没有提供自带的分析工具用于实现对系统状态的穷举搜索,限制了对协议模型的验证能力。

     通过Maude语言对PKMv3协议进行形式化建模,可以实现用户自定义的抽象程度,能够概括描述协议的执行流程,也能够具体定义密钥生成算法。Maude描述的协议规范,简单易懂,能够准确定义协议性质,并且具有可执行性便于实现协议中各主体消息传输的模拟仿真。通过Maude提供的自动化分析工具,能够尽快发现协议设计中的漏洞,从而提高协议设计的安全性\cite{吴昌2008网络安全协议的高效分析系统}。


\section{国内外研究现状}

由于移动WiMAX网络比传统无线网络面临更多的威胁,针对密钥管理PKM协议的安全特性已有学者做了大量的研究工作。以下文献通过形式化的分析与建模方法实现对PKM三代密钥管理协议安全特性的验证。文献\cite{Xu2006Attacks}利用BAN逻辑实现了对前两代密钥管理协议的分析,指出PKMv1协议实施单向认证的缺陷以及PKMv2协议可能遭遇到的穿插攻击。文献\cite{Kahya2012Formal}使用自动化协议检测工具Scyther实现对PKMv2协议的形式化分析,验证出协议中机密性、认证性和完整性的缺陷。针对Scyther验证得出的协议缺陷,作者在文献\cite{Kahya2012Secure}中提出了一种加入CA认证的安全密钥管理协议实现对PKMv2的改进。文献\cite{You2010Verification}在PKMv2协议的基础上通过加入可信第三方实现站点间的双向认证,并通过SSM工具对改进后协议的安全性进行验证。

  文献\cite{Rai2011An}使用一种称作AVISPA的安全协议自动化按钮工具实现对前两代密钥管理协议的验证,发现协议容易遭受到重放攻击。文献\cite{Xu2008Modeling}通过Casper协议建模工具对PKMv2协议进行描述,并用FDR工具分析进程通信的输出结果,发现入侵者可以截获通信消息进行重放攻击。文献\cite{Raju2010Formal}同样使用CasperFDR 工具对PKMv3协议中基站、手机站和中继站间传递明文消息的过程进行建模,分析输出结果检测出窃取密钥的攻击方式。文献\cite{Zhu2015Formal}利用PROMELA语言对PKMv3协议密钥生命周期的时间特性进行建模,并利用DT-Spin工具实现了对协议活跃性,连续性和消息一致性等时间特性的模型检测。

     与此同时,也存在不少学者对WiMAX网络中存在的安全问题进行理论推导和文献综述,分析与对比密钥管理协议中出现的安全漏洞。文献\cite{sikkens2008security}通过文献综述的方式,论述了WiMAX中认证和授权的安全问题,如可能遭遇到的DoS攻击,存在认证缺陷和密钥空间不足等。文献\cite{han2008analysis}给出基于IEEE802.16e标准的移动WiMAX网络中安全框架的综述,经调查发现移动WiMAX网络易遭受到中间人攻击和DoS攻击,因此提出了基于DH公钥密码交换的SINEP协议以提高网络入口的安全级别。文献\cite{Bogdanoski2008IEEE}对WiMAX安全子层的结构框架以及密钥管理协议的基本执行原理进行概述,并总结出WiMAX能够提供强健的用户认证,访问控制,保护数据隐私性以及完整性,但消息的传输仍然存在阻塞,被窃取以及篡改等重大安全威胁。文献\cite{Tshering2011A}描述了WiMAX中的安全机制,同时详细列举了各代密钥管理协议中存在的安全缺陷,包括易遭受DoS攻击,密钥空间攻击,降级攻击等问题,并且分析总结了目前研究者们提出的相应解决方案进行优缺点对比。文献\cite{Tian2007Key}总结了IEEE802.16e中密钥管理协议安全协商,双向认证,密钥生成等过程,同时指出为相同安全组件维护不同的传输密钥状态机的时间和空间上的浪费。文献\cite{杨玖宏2012IEEE802}对802.16m标准中的PKMv3协议进行了安全分析,指出协议中存在无法实现快速切换,并且忽略考虑了中继接入问题,以及挑战消息缺乏安全性保护,并且针对各问题提出了相应的解决方案。

与此同时,针对历代密钥管理协议中存在的缺陷,学者们提出不少改进方案。文献\cite{Hashmi2009Improved}提出一种改进的安全网络认证协议ISNAP,在认证消息中采用同时使用随机值和时间戳的方式以防御网络中的重放攻击和DoS攻击。文献\cite{Altaf2008Security}提出尽管PKMv2弥补了PKMv1协议中的很多安全漏洞,但仍会遭受到重放攻击,并提出与上文类似的随机值与时间戳混合实施保护协议的安全措施。文献\cite{Altaf2008A}针对网络中的非法订阅者发送大量请求消息造成基站计算负担的问题,采用可视秘密共享技术以防御DoS攻击。文献\cite{Kim2008Shared}提出了基于消息验证码的共享认证消息技术,预防网络中的DDoS攻击。文献\cite{Fu2010A}探讨了利用标签ticket实现目标基站间快速切换的机制,减少了重认证的时间开销。文献\cite{Shrestha2009Seamless}提出了一种无缝实时的交接算法,描述了SS在不同BS间的快速以及稳定的切换方式。

     随着形式化分析方法的广泛应用,不少学者通过基于重写逻辑的形式化建模语言Maude实现了对各协议系统的建模。文献\cite{Pita2015Specifying}描述了基于端到端分布式哈希表的Kademlia协议,并通过Maude实现对端到端协议系统的抽象与建模,实验模型将不同行为所花费的时间实现参数化,并且通过时间约束的模型检测以及状态搜索工具证明协议各场景下的性质。文献\cite{Rodr2007The}通过建立连接Maude进程的基础架构,实现通信结点直接向邻居结点或向所有邻居节点组播消息的解决方案,从而在此架构之上实施EIGRP协议,允许Maude进程中运行的对象能够向更远处发送消息。Olveczky和Mesegeur将Maude语言拓展为Real-Time Maude工具,从而用于描述和分析实时和混合系统。文献\cite{Olveczky2008Formal}提出利用Real-time maude形式化模拟,仿真,模型检测实施共享协议的基本技术,并展示了通过该基本技术实例化系统模型,实现对优先级继承协议的验证。文献\cite{Lveczky2001Specification}描述利用Maude形式化方法实现网络组播协议组成中的AER/NCA套件的规格说明和分析验证,Real-Time Maude被证明十分适用于描述协议系统的成分组合,及其时间敏感性和资源敏感性的行为。文献\cite{Lveczky2007Formal}通过Real-Time Maude实现无线传感网络的OGDC算法的建模,建立的无线网络模型克服描述新的通信形式,地理区域的处理,时间依赖以及概率特性等方面的挑战,同时分析验证了系统的正确性以及处理性能。

     本文采用基于重写逻辑的建模语言Maude,对引入时间机制以及入侵者模型的PKMv3协议模型进行形式化分析和验证。Maude建模语言能够用简洁的语义准确描述网络安全协议的对象属性以及消息传输过程,从而建立可执行的形式化模型,并提供用于LTL模型检测和状态空间搜索工具实现对PKMv3协议时间以及安全相关特性的验证与改进\cite{Clavel2008Maude}。

\section{研究内容和方法}

本文的研究内容为IEEE802.16m规范安全子层中定义的PKMv3密钥管理协议的执行过程和基本性质。PKMv3协议实现了WiMAX无线网络中基站与手机端间通过EAP认证方式的相互认证,从而生成授权密钥建立相互连接,并通过三次握手交换安全组件,实现安全参数协商,进而实施密钥材料传输,最终生成传输密钥用于加密后续通信,实现安全信道消息传输。此外,PKMv3协议中的授权密钥与传输密钥都具有有限的生命时长,并且传输密钥的生命周期嵌套在授权密钥中。协议主体需要不断执行相应密钥重认证过程,保证双方不间断的通信过程。由此,本文针对PKMv3整个生命周期的迭代过程进行时间性相关与安全性相关的详细分析。

本文采用形式化建模的方式实现对PKMv3协议的分析与验证。Maude是基于重写逻辑的形式化规范语言,能够实现对分布式系统的建模与验证。将协议系统中的各部分主体的结构和行为分别建模,最终实现系统全局状态在各主体间信息交互过程中的转换。Maude对通信协议进行形式化建模与分析的优势在于它能够对协议系统实现准确的定义,其描述的协议代码简单清晰,十分易用。同时,用户能够决定系统的抽象程度,并通过可执行的形式化规范模拟系统运行。并且提供模拟仿真,系统状态检测,LTL模型检测以及用户自定义的检测工具快速发现协议系统中的漏洞。Maude同时还能够以更小的代价修改协议系统中的错误,并且通过二次建模验证改进后协议的完备性。

本文提出通过形式化规范语言Maude对PKMv3协议整个生命周期的建模方法与验证方式。将PKMv3协议环境抽象为一个系统,系统中包括发送和接受消息的诚实主体和其他入侵者,以及主体间消息传输与入侵者实施攻击的动态行为。对PKMv3协议系统的建模分为以下四个方面:

\paragraph{协议抽象数据类型}
通过Maude中的函数模块实现协议中各抽象数据类型的定义,包括站点类型,证书类型,消息类型,密钥类型等。并通过操作符定义各数据类型代表的对象所具有的性质。

\paragraph{协议网络状态组成}
通过Maude中的函数模块定义协议网络环境的静态组成,包括由主体以及信息集构成的网络节点,网络消息池中的消息组成,以及站点间维护的用于显示密钥生命周期的连接状态。

\paragraph{协议消息传输}
通过Maude系统模块中的重写规则描述协议各执行阶段消息的传输,实现系统状态的动态转换。详细模拟了PKMv3协议中双向认证,安全组件协商,传输密钥生成,加密消息传输以及密钥重认证过程。

\paragraph{入侵者行为}
通过Maude系统模块中的重写规则描述入侵者对诚实主体间会话实施攻击的行为。定义入侵者在协议执行的不同阶段,实行消息截获,重放旧消息,以及伪造新消息以阻断正常通信过程的具体实施方式。

     基于以上建立的可执行协议规范,为其定义初始状态作为执行实例,初始状态中包括具体的站点,传输消息以及连接状态参数。通过Maude中自带的验证方式对PKMv3协议模型进行以下两个方面性质的验证:

\paragraph{时间特性}
 通过Maude提供的LTL模型检测工具实现对协议基本执行过程的连续性,认证密钥以及传输密钥的重认证性等时间相关性质进行验证。

\paragraph{安全特性} 通过Maude提供的状态空间搜索指令实现对协议中消息传输完整性,计算资源可用性以及密钥机密性等安全相关的性质进行验证。


\section{论文结构}

本文一共分为七个章节:

第一章描述了本文的研究背景,简略介绍密钥管理PKMv3协议的安全职能,并引出形式化建模语言Maude,说明通过Maude实现PKMv3协议建模的优势。同时介绍相关课题国内外研究现状,以及本文针对PKMv3协议的具体研究方法。

第二章对Maude语言进行了的详细介绍,包括它的理论基础重写逻辑,Maude函数模块以及系统模块的语法描述,以及Maude所提供的各检测工具的使用方法。同时通过具体的代码示例说明Maude规范的编写和验证方式。

第三章针对IEEE802.16m规范安全子层中的密钥管理PKMv3协议,具体描述其网络层次以及拓扑结构,并详细分析PKMv3协议中的双向认证阶段,安全组件协商阶段,传输密钥生成阶段,加密消息传输阶段以及重认证阶段中基站与手机端交换信息的具体执行过程。

第四章通过Maude语言实现对PKMv3协议各执行阶段的建模。通过Maude中的函数模块定义协议中的各抽象数据类型以及系统状态,通过系统模块中的重写规则描述站点间的消息传输,实现系统状态转换。在模型中引入时间机制,实现密钥生命周期的迭代。同时,通过重写规则描述网络中入侵者的攻击行为。

第五章基于已建立的PKMv3协议模型,利用Maude提供的系统状态搜索指令以及LTL模型检测工具对PKMv3协议模型中的时间特性以及安全特性进行形式化验证,并且总结与分析本文验证结果与他人工作的异同。

第六章针对PKMv3协议中检测出的安全缺陷,提出一种改进后的密钥管理协议,详细定义了协议各执行阶段消息传输的具体内容以及消息保护方式。进而通过Maude实现该安全密钥管理协议的形式化分析与验证,得以证明本文提出的改进方案能够成功弥补PKMv3协议安全漏洞。

第七章用于总结全文工作,首先总结了本文的研究内容,主要贡献,文章所用的实验方法以及验证结果。其次分析了本文实现方法的不足与局限性,提出替代的 建模和验证方式,以及未来可以进行拓展的研究方向。
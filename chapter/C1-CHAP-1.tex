\chapter{绪\hskip 0.4cm 论}


\section{研究背景}
移动机器人是一种集环境感知、智能决策与规划、行为控制与执行等多种功能于一体的综合系统。它集中了传感器技术、信息处理、电子工程、计算机工程、自动化控制工程以及人工智能等多学科的研究成果,代表机电一体化的最高成就,是目前科学技术发展最活跃的领域之一。随着机器人性能不断地完善,移动机器人的应用范围大为扩展,不仅在工业、农业、医疗、服务等行业中得到广泛的应用,而且在城市安全、国防和空间探测领域等有害与危险工作环境中得到很好的应用。因此,移动机器人技术已经得到世界各国的普遍关注。

自主移动机器人的研究开始于上个世纪六十年代末期。斯坦福大学研究院(SRI)的查尔斯·罗森(Charles Rosen) 等人,在1966年到1972年终研发出了命名为Shakey的自主移动机器人。这台机器人被用于人工智能技术的研究,包括复杂环境下机器人的自主逻辑判断、移动路径规划和移动控制。同时,简易的操作式步行机器人也问世,操作式步行机器人步行系统方面的研究是当时的研究热点,解决机器人通过不平整地区的运动不稳定的问题,随后研发和制造拥有多足步行机器人。上个世纪七十年代末,随着计算机嵌入式系统和传感器技术的飞速发展,又迎来了自主移动机器人的研发新高潮。特别是八十年代中期,机器人的研发和设计浪潮席卷全球,当时世界一些著名的科学技术公司着手研制机器人。自从九十年代以来,随着高精度地理环境信息传感器、高速中央处理器、嵌入式控制技术等达到较高水平,以真实物理环境下路径规划技术为标志,对移动机器人开展了更深层次的研究。

许多应用设想在没有中央调度结构的情况下,移动机器人通过自主组织和相互协作,共同完成任务,包括地图的构建、环境检测、城市搜索救援、表面清理、危险区域监控、未知空间的探索等等。在这类的应用中移动算法是保证机器人共同完成任务的关键,意味着移动算法需要进行正式和详尽的验证。在本文中主要关注机器人空间巡逻问题,机器人空间巡逻问题包括巡视一个区域以便检查或保护它,并被广泛的应用于军事、民事领域和一些特殊的系统中。它们可以通过所谓的勘探协议来解决,勘探协议有两种变体,探索终止(exploration with stop)和永恒探索(perpetual exclusive exploration), 分别在文献【f1】和【f1】被提出来。对于探索终止是空间中所有的位置被探索之后,到达某个状态所有机器人永远停止移动。永恒探索是空间中所有机器人都对空间中的每个位置进行反复的访问,永不停止。




\section{国内外研究现状}


\section{研究内容和方法}


\section{论文结构}
本文结构具体如下:

第一章介绍了自主移动机器人空间探索问题在当前和未来实际应用中的重要性和该领域取得的成果与发展。自主移动机器人空间探索算法的核心问题是根据具体的的物理空间如何定义自主移动机器人的行为以保证其完成预设的任务.针对该问题,目前主要使用手动推演与模型检测技术,以验证自主移动机器人算法或协议满足一定的性质。对于稍微复杂一点的移动算法或者移动协议,手动推演的方式,不仅过程冗长复杂,而且容易出现错误或者疏漏。尤其是异步模型调度的情况下,手动推演方法根本就无法进行验证。模型检测技术具备自动性、严谨性和高效性,逐渐被用于各种自主移动机器人算法的验证中。并引出nuXmv实现符号模型检测的方法用于自主移动机器人探索算法或协议的建模与验证。

第二章对nuXmv语言进行详细介绍,包括nuXmv的基础语法、表达式、几种常用的关键字,模块组件、异步系统模型。并对nuXmv中线性时序逻辑(LTL)模型检测进行了介绍。详细介绍了线性时序逻辑(LTL)在nuXmv中的使用。

第三章以自主机器人永恒空间探索协议为例,详细描述在环形空间模型中机器人、空间路径、位置结点等的数学定义。在此基础上,使用nuXmv对自主移动机器人算法或协议在三种不同调度策略下模型的构建。具体介绍了使用LTL公式描述永恒探索性。


第四章本章在模型创建的基础上,分别在三种调度策略的情况下,利用 nuXmv 提供的基于BDD方法及SMT方法实现了机器人探索协议的符号模型检测,验证不同机器人数和图结点数下,协议是否满足永恒探索的需求.当出现不满足情况时,nuXmv 给出不满足的状态路径作为反例,分析性质不被满足的具体原因.

第五章对全文工作做了总结。对实验方法和验证结果进行了分析,总结本文的主要贡献。讨论了nuXmv符号模型检测在空间探索协议验证的BDD方法和SMT方法的选择性,指出了本文工作中的一些不足点,对未来自主机器人领域的研究内容进行了展望。




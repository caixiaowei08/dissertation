\chapter{绪\hskip 0.4cm 论}

\section{研究背景}
移动机器人是一种集环境感知、智能决策与规划、行为控制与执行等多种功能于一体的综合系统。它集中了传感器技术、信息处理、电子工程、计算机工程、自动化控制工程以及人工智能等多学科的研究成果,代表机电一体化的最高成就,是目前科学技术发展最活跃的领域之一。随着机器人性能不断地完善,移动机器人的应用范围大为扩展,不仅在工业、农业、医疗、服务等行业中得到广泛的应用,而且在城市安全、国防和空间探测领域等有害与危险工作环境中得到很好的应用。因此,移动机器人技术已经得到世界各国的普遍关注。

自主移动机器人的研究开始于上个世纪六十年代末期。斯坦福大学研究院(SRI,Stanford Research Institute)的尼尔斯·尼尔森和查尔斯·罗森(Charles Rosen) 等人,在1966年至1972年研发出了命名为Shakey的自主移动机器人。这台机器人被用于人工智能技术的研究,包括复杂环境下机器人的自主逻辑判断、移动路径规划和移动控制。同时,简易的操作式步行机器人也问世,操作式步行机器人步行系统方面的研究是当时的研究热点,解决机器人通过不平整地区的运动不稳定的问题,随后研发和制造拥有多足步行机器人。上个世纪七十年代末,随着计算机嵌入式系统和传感器技术的飞速发展,又迎来了自主移动机器人的研发新高潮。特别是八十年代中期,机器人的研发和设计浪潮席卷全球,当时世界一些著名的科学技术公司着手研制机器人。自从九十年代以来,随着高精度地理环境信息传感器、高速中央处理器、嵌入式控制技术等达到较高水平,以真实物理环境下路径规划技术为标志,对移动机器人开展了更深层次的研究。

许多应用设想在没有中央调度结构的情况下,移动机器人通过自主组织和相互协作,共同完成任务,包括地图的构建、环境检测、城市搜索救援、表面清理、危险区域监控、未知空间的探索等等。在这类的应用中移动算法是保证机器人共同完成任务的关键,意味着移动算法需要进行正式和详尽的验证。在本文中主要关注机器人空间巡逻问题,机器人空间巡逻问题包括巡视一个区域以便检查或保护它,并被广泛的应用于军事、民事领域和一些特殊的系统中。它们可以通过所谓的勘探协议来解决,勘探协议有两种变体,探索终止(exploration with stop)和永恒探索(perpetual exclusive exploration), 分别在文献【f1】和【f1】 被提出来。对于探索终止是空间中所有的位置被探索之后,到达某个状态所有机器人永远停止移动。永恒探索是空间中所有机器人都对空间中的每个位置进行反复的访问,永不停止。

移动空间模型由原始的连续二维欧几里德空间模型,逐渐演化成为有限位置的离散空间模型。离散空间使用图来描述,图的结点代表空间中机器人可以到达的位置,图的边表示机器人可以由一个位置结点到达相邻的结点的路径。这样塑造空间模型更加简便直观,更加关注机器人和空间位置结点数量关系对移动算法设定的影响。离散空间模型中,每个位置结点在同一时刻至多只有一个机器人占据。机器人有以下三个特点:1、机器人无记忆存储器,即机器人无法记录历史移动过程;2、无方向传感器,机器人在离散空间中,无法辨别方向,并且没有方向偏好。3、无通讯功能,即机器人之间无法接收或者发送消息;4、 拥有视觉传感器,通过视觉传感器机器人可以获取空间中其他机器人的位置信息;5、运算单元,处理移动算法匹配问题,通过计算给机器人下达移动指令,控制机器人的移动;6、移动装置,通过移动装置,机器人可以沿着可移动的路径做出移动动作;7、 匿名性,机器人无法通过无表进行区分,即所有机器人都是相同的,且在同一个离散空间系统中,所有机器人执行相同的移动算法。

设计机器人移动算法,并保证机器人设定移动算法之后,能满足探索终止或者永恒探索是目前机器人巡逻问题的一个研究热点。机器人移动算法的设计和验证,大多是以手动推演的方法为主。手动推演不仅过程冗长复杂,在推演的过程中也容易出现错误或者疏漏,尤其是在某些特定的设置条件下,手动推演方式根本无法进行验证。形式化验证方法因其自动性、高效性、严谨性逐渐被应用于各种移动机器人算法的验证。

nuXmv是一个新的形式化的符号模型检测工具,不仅集成了最新的SAT、BDD模型检测的算法,而且还包含验证性能高效的基于SMT的验证技术。使用nuXmv工具建模验证时,可以根据实际需求选择验证方式,十分灵活。nuXmv 符号模型检测方法不仅不依赖于某个具体的初始化状态进行验证,而且模块化建模复用性高,大大简化了建模的复杂性。符号模型检测的高效性可以有效避免验证过程中状态爆炸问题。

\section{国内外研究现状}
自主移动机器人领域主要的研究方向有导航和定位、路径规划、多传感器信息融合技术、多机器人系统、机器人仿生技术。影响自主移动机器人研究发展的因素包括传感器技术、运动控制策略、通讯技术、导航和定位系统。进入21世纪,随着计算机嵌入式系统和传感器技术的飞速发展,能够供机器人用的高精度传感器不断研发和生产,计算机运算处理能力显著提升,机器人技术已经取得巨大的进展,研究成果硕果累累,然而还远没有满足人类对机器人的需求。

自主移动机器人早期的代表是斯坦福大学研究院(SRI,Stanford University Research Institute)研发取名叫Shakey的机器人,用于人工智能(AI,Artificial Intelligence)技术应用的研究。随着机器人技术的进步,研制出各式各样的移动机器人,按照工作环境来分可以分为室内机器人和室外机器人;按照机器人移动方式来分有轮式机器人人、步行机器人、履带式机器人等;按作业空间区分有水下机器人、陆地机器人、无人机、空间机器人等;

本文主要研究是机器人的空间巡查问题,


Lelia Blin在文献【ring】中提出了环形离散空间中,满足永恒探索的最小移动算法和最大移动算法,并按照算法的设定,使用手动推演的方式证明了两种算法的正确性。为了提升移动算法验证的准确性、高效性,并能降低验证成本,Béatrice Bérard等人使用DVE形式化语言对机器人移动算法进行建模,通过DiVinE和ITS验证工具,实现了移动算法的验证。不仅如此,北陆先端科学技术大学院大学(JAIST)信息科学学院的Ha Thi Thu Doan使用形式化逻辑重写语言Maude对


\section{研究内容和方法}
本文的主要研究内容是机器人巡逻问题中自主移动机器人永恒探索算法的形式化方法的改进。机器人移动算法是自主移动机器人协作完成永恒探索任务的核心。空间中所有机器人执行相同移动算法,在没有任何外部控制器的情况下,通过视觉传感器获取的其他机器人的位置信息,与机器人移动算法进行匹配,做出移动决策。机器人不断反复获取空间位置信息、匹配移动算法、做出移动操作,使得每个机器人对离散空间模型中每个位置结点进行反复的访问,称为永恒探索。满足永恒探索的移动算法称为永恒探索算法。本文针对离散空间中机器人在不同的调度模型下移动过程的详细形式化描述和分析。

本文采用符号模型检测的方法实现对机器人移动算法建模,验证移动算法是否是永恒探索算法。nuXmv是新的符号模型检测工具,能胜任离散空间移动算法的建模验证与分析。对机器人、离散空间、移动算法分别进行模块化建模,模块化代码的可移植性很好,简单清晰,十分方便。完全同步调度模型、半同步调度模型、完全异步调度模型中,模块化代码可以复用,使得复杂的编码过程变得容易,只需针对不同调度模型做出一些对应调整。使用LTL公式分别描述永恒探索算法所需满足的非冲撞性、非互换性、非终止性,并可以通过nuXmv分别进行验证,在不满足任何一条性质时,nuXmv可以给出反例状态路径,便于原因分析。为了进一步提升移动算法验证的效率,nuXmv验证代码通过代码生成器进行生成,只需录入移动算法序列、机器人数等相关信息。

\section{论文结构}
全文一共分为五个章节:

第一章介绍了文章的研究背景,简单描述了自主移动机器人永恒探索算法,并引出了形式化符号模型检测工具nuXmv,说明使用nuXmv工具实现永恒探索算法建模的优点。国内外相关课题的研究现状,以及本文使用nuXmv对永恒探索算法的形式化建模验证的方法。

第二章对nuXmv工具进行详细介绍,包括符号化建模适用范围,有限状态机的定义,模块化建模,CTL和LTL模型检测命令。通过具体的代码示例描述nuXmv的建模和验证过程。

第三章具体介绍探索空间的可能的拓扑结构,详细描述了机器人,探索空间、移动算法的具体内容给出了确切的定义。具体分析自主移动机器人永恒探索算法,包括机器人的移动阶段,机器人移动算法匹配过程,永恒探索的性质。提出三种移动机器人的调度模型,并分别调度模型进行了详细介绍。

第四章提出使用nuXmv符号模型检测方法,对自主移动机器人永恒探索算法进行建模验证和分析。以环形拓扑空间的最小移动算法为例,构建最小移动算法在三种不同的调度模型下的模型,并采用nuXmv中的基于BDD算法和SMT的验证指令,进行了形式化的验证,对实验结果也进行详细的分析。

第五章对全文工作做了总结。对实验方法和验证结果进行了分析,总结本文的主要贡献。讨论了nuXmv符号模型检测在空间探索协议验证的BDD方法和SMT方法的选择性,指出了本文工作中的一些不足点,对未来自主机器人领域的研究内容进行了展望。

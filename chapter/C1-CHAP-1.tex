\chapter{绪\hskip 0.4cm 论}


\section{研究背景与意义}


\section{国内外研究现状}


\section{研究内容和方法}


\section{论文结构}
本文结构具体如下:

第一章介绍了自主移动机器人空间探索问题在当前和未来实际应用中的重要性和该领域取得的成果与发展。自主移动机器人空间探索算法的核心问题是根据具体的的物理空间如何定义自主移动机器人的行为以保证其完成预设的任务.针对该问题,目前主要使用手动推演与模型检测技术,以验证自主移动机器人算法或协议满足一定的性质。对于稍微复杂一点的移动算法或者移动协议,手动推演的方式,不仅过程冗长复杂,而且容易出现错误或者疏漏。尤其是异步模型调度的情况下,手动推演方法根本就无法进行验证。模型检测技术具备自动性、严谨性和高效性,逐渐被用于各种自主移动机器人算法的验证中。并引出nuXmv实现符号模型检测的方法用于自主移动机器人探索算法或协议的建模与验证。

第二章对nuXmv语言进行详细介绍,包括nuXmv的基础语法、表达式、几种常用的关键字,模块组件、异步系统模型。并对nuXmv中线性时序逻辑(LTL)模型检测进行了介绍。详细介绍了线性时序逻辑(LTL)在nuXmv中的使用。

第三章以自主机器人永恒空间探索协议为例,详细描述在环形空间模型中机器人、空间路径、位置结点等的数学定义。在此基础上,使用nuXmv对自主移动机器人算法或协议在三种不同调度策略下模型的构建。具体介绍了使用LTL公式描述永恒探索性。


第四章本章在模型创建的基础上,分别在三种调度策略的情况下,利用 nuXmv 提供的基于BDD方法及SMT方法实现了机器人探索协议的符号模型检测,验证不同机器人数和图结点数下,协议是否满足永恒探索的需求.当出现不满足情况时,nuXmv 给出不满足的状态路径作为反例,分析性质不被满足的具体原因.

第五章对全文工作做了总结。对实验方法和验证结果进行了分析,总结本文的主要贡献。讨论了nuXmv符号模型检测在空间探索协议验证的BDD方法和SMT方法的选择性,指出了本文工作中的一些不足点,对未来自主机器人领域的研究内容进行了展望。




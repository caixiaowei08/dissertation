\chapter{总结和展望}

\section{总结}
本文使用nuXmv符号模型验证工具对自主移动机器人空间永恒探索算法进行建模与验证,对实验结果进行分析。使用参数模块定义了机器人模型,LTL描述永恒探索的性质。通过使用nuXmv工具对最小移动算法的验证,提出了一种对空间永恒探索算法的验证方法。

本文中主要的工作包括以下几点:

第一,本文提出了机器人符号化建模方式,使用参数模块定义机器人模型,对移动算法及其匹配进行了描述。同时,使用LTL对空间永恒探索性质也进行了定义。

第二,本文在完全同步调度策略、半同步调度策略和完全异步调度策略下,实现了空间探索算法的模型构建,并使用BDD和SMT验证方式,分别验证模型的永恒探索性质。

第三,本文在已有自主移动机器人用探索算法形式化验证的基础上,实现了符号化模型的验证与分析。不仅提升的机器人移动算法的验证效率,而且扩展了验证功能。能够在任意机器人初始位置时,验证移动算法性质。

\section{展望}
本文主要是研究机器人空间探索问题。目前在使用形式化方法对机器人移动算法进行验证的研究工作并不是很多。本文也只是针对机器人空间探索问题的一小部分进行了研究。

机器人领域的研究十分广泛,除了空间探索之外,还有仿生学、网络机器人、多机器人系统等等。目前机器人领域的研究已经取得了很多成果,但是在某些方面离实用要求还是有些差距。随着传感器技术、机器学习和嵌入式技术的不断发展,机器人在未来人们生活和生产中逐渐扮演重要角色。

对于自主移动机器人空间探索问题,使用符号模型检测方法可以有效的避免状态爆炸问题,然而验证随着空间位置结点和机器人数增加,符号模型的验证的时间也在快速增长。本文提出的符号模型验证方法,目前还不能直接用于实际系统中空间结点或机器人结点较多的应用场景。在符号模型验证方法的基础上,通过抽象或者归纳技术,如nuXmv内置的k-induction方法,解决验证效率问题,是未来需要进一步深入研究的工作。


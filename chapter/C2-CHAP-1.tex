
\chapter{nuXmv的介绍}
nuXmv是一种新的符号模型检测器,支持有限状态系统和无限状态系统的建模分析。nuXmv继承NuSMV,支持所有的NuSMV 的功能。在NuSMV功能基础上,不仅在限状态系统和无限状态系统两个方面增加了新的特性,还在其他方面进行了提升。NuSMV是在一种基于BDD算法实现的SMV 验证器上进行重新实现和功能拓展所得出验证工具。NuSMV 被设计成一种开放式的模型检测框架,作为一个高可信的验证工具被广泛的应用于工业设计以及其他研究领域。在NuSMV2版本中,功能得到进一步提升,不仅具备原来由科罗拉多大学研发的BDD算法模型检测组件,而且添加了SAT 模型检测组件,SAT模型检测组件中包含基于RBC的边界模型验证器。到了2.5.0版本,热那亚大学为NuSMV 捐献了SIM,SIM中包含最新的SAT算法解析器,其中的RBC 软件包支持边界模型检测算法。

同NuSMV相比较,nuXmv为状态和输入变量新增两种数据类型实数(real)和整数(integer),这使用户能够对无限状态转换系统的规范进行建模。添加了新的结构,用于指定抽象技术中谓词。nuXmv不仅支持所有的NuSMV交互式命令,而且添加一些新特性对应的新的操作命令,应用于有限状态转移系统中新的模型检测算法和无限状态转移系统中基于SMT 的新的检测算法。

nuXmv支持计算树逻辑(CTL)和线性时态逻辑(LTL),在nuXmv代码中使用CTL或者LTL公式描述初始性质和验证性质,使得代码比较简洁、具备更强的表达能力。在无限状态系统验证过程中使用基于SMT的验证方法,大大提升验证的效率,特别是在验证不满足性时,可以高效的求得不满足的反例。nuXmv不仅可以构建同步系统模型,而且可以构建异步系统模型,满足一些系统建模的需求。由于nuXmv 支持多种建模验证方式,用户可以按照自己的需求进行选择。

\section{}



\section{本章小结}